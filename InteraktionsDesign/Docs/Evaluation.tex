\section{User Evaluation}
\subsection{Description}

\textbf{Introduction and hypotheses}\\
The userinterface has been designed to look similar to current forum-style "social media". This choice of design was made such that the average user would be facing as little a hurdle as possible, in transitioning from their current forum of choice to this product. To test if this is actually the case the following \textit{null hypothesis} is posed:\newline
\textit{- Similar designs will make transitioning difficult for users}

\noindent
\\\textbf{Subjects}\\
%Indicate who your subjects are, how you obtained them, what incentives (if any) were provided, any relevant demographics, how they were divided into categories, and so on.
The test subjects in the experiment were found at the university, in or close to our own hall. This means that the subjects in the test were all of rather similar educational programs. This would tend to cause the test subjects to be somewhat similar to our selves in mindset. The test subjects are also probably of a higher than average technical skill level and might be more familiar with forums than the average user.

\noindent
\\\textbf{Materials}\\
The prototype is run as a simple interactive website controlled from a browser on a computer. The users use the computer's keyboard and mouse to navigate the prototype. As the prototype is programmed only using HTML, CSS and JavaScript there is no need for extra software other than the web browser.

\noindent
\\\textbf{Methods}\\
The experiment will run in two stages: The first stage is using the qualitative method "think aloud", where the user is asked some questions while navigating the prototype. These questions are asked on the fly. The second stage is using the quantitative method "System Usablity Scale (SUS)", which is an acknowledged questionnaire, that measures how the usability of the system was. Through these to methods the results can be triangulated to provide more certainty in the experiment.

%Describe the method employed to run your experiment. This should include the experimental design, variables examined, how data was collected, conditions for running the subject, and so on.
\noindent
\\\textbf{Problems}\\
%Describe any problems/limitations encountered that will help other researchers avoid or account for them if they decide to replicate your experiment.

\subsection{Results}

%The System Usability Scale (SUS)
%https://www.moodle.aau.dk/pluginfile.php/1289446/mod_resource/content/1/ID8.pdf
%slide 27

Test
Tænke højt test?
- Navigation, hvordan er det at finde rundt i?
- Sikkerhed, 
    Hvor klart er det at man er sikker? 
    Hvordan kunne det blive præsenteret bedre?
\\
Report the data you collected: quantitative (e.g. means and if so than standard deviation are a must here) as well as qualitative data\\
If you have, include level of statistical significance (not a must)\\
Graphs, Plots, Histograms etc. underline your data through figures\\
Only report critical raw data in this section, do not start to interpret yet here!

\subsection{Discussion}

Interpretation of results (what do you think they actually mean?)\\
Relation to other work (is it somewhat coherent to results of others?)\\
Critical reflection\\
-Criticize your work (constructively!)  indicate possible flaws, mitigating circumstances, the limits to generalization, conditions under which you would expect your findings to be reversed, and so on.
Future Work\\
-What do you need to redesign/change?\\
-What should be done next?