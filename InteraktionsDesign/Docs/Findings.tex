\section{Questionnaire findings}
The group has found, from the 19 responses, that in general about 4 different personas can be found, moreover two further personas exist in different branches. These personas have been separated by whether or not they have actively tried to enhance their own personal security, and their evaluation of how much they feel their privacy is preserved "versus" how much information they think is available about them selves. Some of the participants have filled in their contact information and confirmed their willingness to participate in follow-up interviews.

\subsection{Persona 1 (The non concerned):}
This is the most common persona, with about 8 of the results.
This persona mostly uses their computer for personal matters or entertainment, are active at 7-8 different social media a day,
and uses their home computer for an average of 6 - 8 hours a day.
They are not concerned what others think of them, and therefore also do not hide their personality on social media.
They feel, to a greater extent, than the other personas, that their privacy is preserved on the internet, though they still know that most of their information is publicly available.
They have searched their own name online, mostly for fun, but really do not care about actively trying to enhance their privacy with use of VPN or the like.\\

\textit{From this Persona exists two branches:}
\begin{itemize}
    \item 
    \textbf{The Knowing:}\\
    This branch is highly aware that all of their personal information is publicly available, mainly because of their extensive use of social media.
    \item 
    \textbf{The Ignorant:}\\
    This branch knows that their personal information is public available, though they still believe that some information would be hard, or impossible, to find on the internet.
\end{itemize}

\subsection{Persona 2 (The concerned):}
This is the second most common persona, with about 6 participants fitting the description.
This persona mostly uses their computer for entertainment or work, is active at 6-7 different social media a day,
and uses their devices for an average of 8-10 hours a day.
They are, in some way, concerned what others think of them, and therefore also, to some extend, try to act a certain way on social media, to seem nice and polite.
They feel, to a lesser extend, that their privacy is not preserved on the internet, and here the two branches also differ the most.
They have searched their own name online, and care about trying to enhance their privacy with use of VPN or the like.\\

\textit{From this Persona exists two highly different branches:}
\begin{itemize}
    \item 
    \textbf{The Successful:}\\
    This branch do not feel that their privacy is preserved, and thereby also actively try to enhance their security. They also feel pretty successful in their endeavour, or is rather ignorant of their failure.
    \item 
    \textbf{The Failed:}\\
    This branch do neither feel that their privacy is preserved, despite their high curiosity, nor do they feel that they have succeeded in their security attempts.
\end{itemize}

\subsection{Persona 3 (The user)}
This persona can not really be classified to a single use, as they use their computer for the sake of necessity for school, work or social matters, also they can be active in any average of time and number of social media a day.
They do not act differently on social media or real life as they do not see differentiate between them.
They range all the way from being very concerned about privacy on the internet, to being perfectly happy about it. What they have in common is that no matter their views they have no intention to actively improve upon their privacy.  
They could have tried searching their name, but either way they really do not know to which extent their information is publicly available.

\subsection{Persona 4 (The Extremes)}
This persona either can not fit another perona due to their extremely conflicting responses, perfectly fit a persona but do so to such an extreme degree that it is over the top, or they could have modified their results to a suspicious extent.