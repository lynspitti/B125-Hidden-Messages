\newpage
\section{Vejledermøde 05-03-2018}
\subsection{Hvad er der sket siden sidst}
    \begin{itemize}
        \item Martin er kommet tilbage
        \item Problemanalyse, init indledning og problemformulering skrevet og afsendt (Sprint 1)
        \item Påbegyndelse af Sprint 2, metode og kravspek.
    \end{itemize}
\subsection{Gennemgang af fremsendte arbejdsblade (Hvis fremsendt)}
    \begin{itemize}
        \item Rigtig meget tekst om gamle dage i problemanalysen. Mere fokuseret, stadig være åben.
        \item Hvordan bruger vi den historiske gennemgang til i projektet?
        \item Mere klarhed på synligheden på den gemte besked, som er vores hovedemne
        \item Stil spørgsmålet: Hvorfor skal vi lære om dette nu? ved hver sektion i problemanalysen
        \item Se på ordet: umiddelbart - I vores problemformulering
        \item "De teknologiske muligheder for at skjule beskeder er større end nogensinde"
        \item Ryk mere historiske gennemgang i introduktionen.
        \item Mere analyse i problemanalysen (kan skrives som delkonklusion under hver afsnit)
        \item Mere simpel usecase diagram, benyt kun en kasse!
        \item Går måske alt for langt tilbage i tiden i analysen.
        \item Historie - Konsekvenser - init problemformulering - Problemanalyse - Problemformulering
        \item Hvad ligger bag grund til afsnittene, samt sammenhængen og rækkefølgen af dem
        \item Alt for meget fokus på kryptering.
        \item Da Pinchi Code er glimerede som indledning / introduktion
        \item "Hvis det er gentagende eller ikke giver noget til rapporten, så slet det"
    \end{itemize}
\subsection{Specifikke spørgsmål}
\subsubsection{Uddannelsens opbygning}
    \begin{itemize}
        \item Kan man forvente, at senere projekter vil tillade, og forvente, mere arbejde på den endelige implementation af produkter?
        \item Hvilken type opgaver er uddannelsen rettet imod? Er den primært fokuseret på processen der leder op til produktet (som vi indtil videre har været fokuseret på), eller vil man senere opnå markant større kompetancer inden for de mere håndværks orienteret dele? 
    \end{itemize}
\subsection{Gennemgå og opdater tidsplan. Gantt og Trello}
    \begin{itemize}
        \item Tid til at rette på problemanalysen i Sprint 2
    \end{itemize}
\subsection{Eventuelt}
    \begin{itemize}
        \item Strejke / lockout
    \end{itemize}
\subsection{Dato og tidspunkt for næste vejledermøde}
    \begin{itemize}
        \item Julie Onsdag 14-03-2018 13:30 - Problemanalyse (Send arbejdsblad fredag, senest mandag)
        \item Rasmus Mandag 19-03-2018 ubestemt tid - Snakke med Rasmus om kravsspek (SSU workshop)
    \end{itemize}