\documentclass[11pt]{article}
\renewcommand{\baselinestretch}{1.20} 
\usepackage[danish]{babel}
\usepackage[utf8]{inputenc}
\usepackage{url}
\usepackage{csquotes}
\usepackage{graphicx}
\usepackage{subcaption}
\usepackage{pdfpages}
\usepackage{caption}
\usepackage{geometry}
\usepackage{float}
\usepackage{mdframed}
\geometry{a4paper,total={170mm,237mm},left=20mm,top=30mm}
\usepackage{fancyhdr}

\pagestyle{fancy}
\fancyhf{}
\rhead{AAU}
\lhead{Gruppe: B-125}
\chead{P2 - Procesanalyse}
\rfoot{Side \thepage}

\begin{document}

% Title page
\begin{titlepage}
    \centering
	\includegraphics[width=0.35\textwidth]{Projectdoc/Assets/Illustrationer/aau_logo_da.pdf}\par\vspace{1cm}
	{\scshape\Large P2 - Procesanalyse\par}
	\vspace{0.2cm}
	{\huge\bfseries Skjulte beskeder i det åbne\par}
	\vspace{0.2cm}
	{\scshape\Large ITC B125\par}
	\vspace{2cm}
	{\Large\itshape 
	    Benjamin Bach Jensen\\
        Daniel Benjamin Vestergaard Jensen\\
    	Mikkel Steen Hansen
    \par}
	\vfill
	\vfill
\end{titlepage}

% Table of contents
\renewcommand{\baselinestretch}{0.8} 
\tableofcontents
\renewcommand{\baselinestretch}{1.20} 
\newpage 

\section{Indledning}
Under P2 forløbet har gruppen, allerede fra start, oplevet megen indhold til proces og forløb. Gruppen har nemlig erhvervet flere nævneværdige erfaringer, strækkende lige fra debat og konflikter, til gode sociale sammenknytninger, og oplevelser for anvendelse af nye værktøjer. Gruppen har oplevet mandefald allerede fra anden dagen, og senere også følelsesmæssigt akavede håndteringer, der desværre resulterende i endnu en svær beslutning om gruppesplittelse.

\section{Projektplanlægning}
\subsection{Beskrivelse}
Gruppen var fra start meget struktureret, med store forventninger, og vi fik hurtigt aftalt en standard for det kommede arbejde. Under denne proces havde gruppen hovedsageligt planlægningsmetoderne scrum og agile i tankerne, og fik derfor udarbejdet den forventede fremgangsmåde som følgende, også indskrevet i projektets gruppekontrakt [Se bilag \ref{gruppekontrakt}].\\
Hverdagen startede fra 08:15 til 16:15, og ville indeholde et morgenmøde 08:30, eller efter eventuelle kurser, hvorefter alle gerne skulle være klar med en plan, for deres resterende arbejdsdag.\\
Alle opgaver skulle planlægges i planlægningsværktøjet Trello, hvor fra en Gantt-oversigt skulle genereres. Ingen opgaver måtte blive markeret "Færdig", før de var blevet reviewet (behandlet) af et andet gruppemedlem. Vi opdelte særlige milepæle i projektet i såkaldte "sprints", som forløb i et spænd af cirka 2 uger hver.\\
I starten af projektetforløbet gik det efter forventningerne, dog med undtagelse af morgenmødet, som aldrig rigtigt blev indlejret. Et møde blev stadig til nogen grad anvendt, men ofte først længere oppe på dagen, eller mere undervejs med dagens arbejdsgang, indtil den til sidst ikke længere blev udført, hvilket også yderligere påvirkede resten af gruppens planlægningsproces. Senere i projektet gik alle ofte blot i gang med en individuel opgave, hvis de ikke allerede havde mistet overblikket. Dog blev resultatet på dette, at nogle tog initiativ, mens andre for det meste havde en vis forvirring. Mange gange sad hver især bare og ventede på, at en ville tage initiativ til at føre an, men dette kunne desvære godt tage sin tid. På den anden side, når det skete at en plan for dagen var blevet bestemt, og skrevet enten på en tavle eller i Trello, blev disse også udført forholdsvis efter de initierende forventninger.

\subsection{Erfaringer og vurdering}
Som førnævnt, oplevede gruppen at det mest bindende led for den tænkte fremgangsmåde var morgenmødet. Da disse møder begyndte at forsvinde, var Trello den første der led under samme. Trello viste sin bedste anvendelse under de første mange opgaver, der faktisk blev registreret, men efterhånden som de var blevet færdiggjort, og rettet for femte gang, blev færre og færre opgaver registreret, og i stedet blev opgaverne bare påbegyndt. Manglen på Trello aktivitet viste sig selvfølgelig også videre i Gantt-oversigten, der nok var det hårdest ramte værktøj, da denne fungerede som et tredje led, og blev lavet ud fra et plugin tilføjet til Trello. Senere i projektet fandt gruppen dog også ud af, at Trello's forskellige plugin faktisk ikke er særligt anvendelige for sådanne skole projekter. Trello giver nemlig kun en af de forskellig plugin's gratis, og her efter skal man betale for de resterende. Dog er det heller ikke nok at betale for pluginet på Trello, da man ofte også skal betale under pluginet's eget hjemmeside, som i de fleste tilfælde koster rigtig mange penge, eller kun tillader at blive anvendt i en mindre periode.\\
Altså viste det sig at den eneste ikke påvirket metode, som også forblev under hele projektet, var "reviews". Dog blev også selv denne tilsidst, efter manglen på møder, ofte til en opgave, der blev kastet hen over bordet, til det gruppemedlem der var mest ivrig efter noget at lave.

\subsection{Refleksion}
Gruppen har af dette forløb lært flere ting, som selvfølgelig også vil blive taget til overvejelse ved næste projekt. Gruppen har fundet review processen meget naturligt, efter første tilvendelse, og Trello værende et generelt godt værktøj. Trello skal dog anvendes fast, for en effektiv effekt, og har derfor tendens til pludseligt at blive glemt, hvis ikke det har en repræsentativ ordføre. Til gengæld er Trello også forholdsvist nemt at genindføre, selv midt under de forskellige forløb, hvis uheld endelig skulle ske. Gantt viste også sin anvendelse for et godt overblik, men er generelt besværligt, hvis det anvendes under Trello, da det både kræver en repræsentativ ordføre, og en større betaling.

\section{Gruppesamarbejde}
\subsection{Beskrivelse}
Gruppen har prøvet at nå til enighed om mange forskellige problematikker. Dette har dog flere gange taget lang tid, samt kun efterladt små mængder af energi til det egentlige arbejde.\\
Som det første fik gruppen dannet en gruppekontrakt. I denne var faste mødetider blevet vedtaget, samt struktur for daglig gang. Morgenmøde, frokostmøde og et bestemt tidspunkt for frokostpause. Ydermere var der også vedtaget ugemøde hver fredag.\\
Selvom gruppekontrakten har lagt nogle linjer for hvordan samarbejdet skulle foregå, så har vi ikke brugt kontrakten til at håndhæve dette. Vi har dog holdt faste mødetider uden problemer alligevel. Til gengæld er strukturen med to møder om dagen, og et længere ugemøde om fredagen, kun blevet anvendt i løbet af starten af projektet. Der har været holdt nogle møder i løbet af resten af projektet, men det har ikke nødvendigvis været dagligt, eller resulteret i klare retningslinjer for resten af tiden indtil næste møde.\\
Det var også aftalt i gruppen, at al tiden afsat til kurser ville blive respekteret, og at alle i gruppen vil hjælpe hinanden under opgaveregning.

\subsection{Erfaringer og vurdering}
Samarbejdet fra dag til dag har ikke fungeret særligt godt, både på grund af manglen på afholdte møder, men også på grund af manglen på enighed til møderne. Vi har nemlig prøvet at nå til enighed i flere situationer, hvor mødet er endt med uenighed om fremgangsmåden. Disse møder har dermed ikke hjulpet på fornuftig udnyttelse af tiden, da resten af dagen ofte har endt i ingen arbejde overhovedet, grundet manglende mentalt overskud.\\
Gruppekontrakten har også været mindre optimal, ikke fordi indholdet har været dårligt, men fordi det ikke har været brugt, samt at kontrakten ikke har været opdateret siden den blev skrevet.\\
Gruppen har haft holdt tiden afsat til opgaveregning, uden at distrahere sig selv med projekt. Vi har i gruppen dog generelt ikke været gode nok, til at spørge hinanden om hjælp til opgaveregningen. Dette var ellers især vigtigt, da vi hverisær er markant bedre til enkelte ting, end resten af gruppen. Af netop denne grund, er det et stort potentiale, som ikke har været udnyttet i særlig høj grad.

\subsection{Refleksion}
For fremtiden vil det være godt, hvis ethvert møde afsluttes med en decideret handlingsplan. Dette skal forekomme uanset (u)enighed i gruppen. I værste fald kan en afstemning være nødvendig. Det vil også være favorabelt at begrænse længden af diskussioner til en fast længde, afhængigt af emnets omfang.\\ 
Angående gruppekontrakten er det tydeligt, at den skal holdes opdateret, for at kunne afspejle ændringerne i gruppen i løbet af projektet. Derudover skal den også bruges meget mere. f.eks. vil det være godt, at bruge den til at skabe fokus i gruppen, når en diskussion har taget for lang tid, eller at en enighed ikke har kunne opnås.\\
Til kursus gangene, og især opgaveregning, er det i fremtiden meget vigtigt, at vi ikke er utilpasse med at spørge om hjælp. Dette vil nemlig hjælpe os alle til at komme meget bedre, og nemmere, igennem studiet som helhed. 

\section{Samarbejde med vejleder}
\subsection{Beskrivelse}
Gruppen havde lagt vægt på at skrive en separat vejlederkontrakt, da tidligere erfaringer med vejlederdelen, som en lille sektion af gruppekontrakten, ikke havde lagt det bedste fundament for samarbejdet. Vejlederkontrakten [Se bilag \ref{vejlederkontrakt}] tilbød mere struktur i møderne, ved at opsætte en konkret rollefordeling med ordstyrer, referent og gantt mester. Ydermere blev en forventning, på cirka et møde om ugen, samt en fast dagsorden indsat. Der var fra start et ønske om et aktivt samarbejde, hvor et møde ikke kun skulle planlægges hvis gruppen oplevede stopklodser, men også som en general vejledning, sådan at gruppen ikke drejede i den forkerte retning under projektet.\\
På P2 havde vi, udover en hovedvejleder, nu også en bivejleder. Hendes speciale lå primært i den indledende del af projektet, og især til problemanalyseringen. Dette var nyt for os, og gav nogle lidt forudsete situationer, f.eks. var bivejlederen ikke inde over vejlederkontrakten, og vi planlagde primært møderne med de to vejledere hver for sig. 

\subsection{Erfaringer og vurdering}
Aftalerne vi indgik i vejlederkontrakten gik knap så godt, jo længere inde i projektet vi kom. Møderne kunne pludselig ikke holdes indefor det ugentlige interval, der ellers var tiltænkt, dels fordi vores vejleder fik travlt med andre ting, og dels fordi vi ikke følte vi havde nyt at vise frem. Presset fra de andre opgaver, opbygget af de forskellige kurser, fik tildels skubbet projektets forventede fremgang til en negativ retning, og derfor også skabt denne følelse af mangel til de individuelle møder. Da vi endelig havde møderne gik det såmænd fint, vi fulgte vores dagsorden, og skrev kort referat. Kommunikationen over mail var dog lidt vanskeligere, da der internt i gruppen blev diskussion om meningen bag tilbagemeldingen. Denne diskussion kunne være løst lettere ved flere fysiske møder omkring meningen bag rettelser og mærkninger til arbejdsbladene.

\subsection{Refleksion}
En af de største muligheder vi ikke udnyttede godt nok, var at benytte vejlederne som en samlet enhed. Dette kunne gøres ved at arrangere flere møder, hvor både hoved- og bivejleder deltager. Denne tilgang har potentialet til at kunne slå to fluer med et smæk, da de samtidig kan supplere hinanden. De to vejleder ville også kunne give hinandens input på deres individuelle ekspert områder, eller agere mere som en hjælpende hånd uden for disse samme områder.\\
Et forslag til at holde en sund aktiv kontakt til vores vejledere, ville være at planlægge faste afleveringer af arbejdsblade. Derved ville vejlederne altid kunne følge med, hjælpe gruppen på rette kurs, og samtidig også sørge for at gruppen ikke lægger projektet for langt tilside, i stedet for f.eks. andre opgaver i kurser. Hvis at arbejdsbladene lagde grund for vejledermøderne, ville en masse misforståelser i feedbacken også være undgået.\\
Med hensyn til vejlederne, som mæglere i gruppesamarbejdet, så var vi ikke gode nok til at aftale ekstra ordinære møder, som der f.eks. kunne have været brug for i henhold til problemerne under et gruppemedlems fravær og senere udsmidning. Hertil kunne begge vejlederes holdninger være yderst brugbare.

\section{Læringsproces}
\subsection{Beskrivelse}
Vi har primært i projektet haft nogle arbejdsopgaver, hvor der var en naturlig ekspert på området, f.eks. i programmering, analyse eller design. Derved har det sjældent været den svageste, der har arbejdet med et givet emne. 
I henhold til projektet, har der kun været små mængder af makkerarbejde, og individuelle arbejdsprocesser har generelt domineret forløbet.\\ 
Det samme har, i en vis udstrækning, gjort sig gældende i forbindelse med kurserne. Her har vi nemlig kun nogle gange arbejdet decideret sammen på opgaveregning. Ellers har vi dog i større omfang spurgt hinanden til råds om opgaverne.\\
Det har oftest forholdt sig sådan, at hvis en forelæser/bog/vejleder har været uforståelig, har vi brugt lang tid på at diskutere det, i stedet for at spørge forelæseren.

\subsection{Erfaringer og vurdering}
Læringsprocessen er desvære mere eller mindre blevet overset under dette forløb, og gruppen har, trods dens lille bemanding, været opbygget af vidt forskellige kompetencer. En eventuel præsentation og gennemgang af de enkeltes læringsstile ville, i bagklogskabens klare lys, derfor nok også have givet et mere favorabelt resultat for alle. Efter denne manglende overensstemmelse, kan vi dog konkludere, at gruppens læringsproces virker mere eller mindre rodet at kigge tilbage på. Nogle ville nok have haft mere ud af de enkelte emner, fag og udvikling, hvis gruppen havde gået mere op i konkret kammeratskabslæring.\\
Gruppen kan dog konkludere, grundet det større tidspres, de manglende ressourcer, og det førnævnte brede kompetencer, at gruppen for det meste nok nærmere har anvendt en struktur, hvor læringen var op til den enkelte. Gruppen har selvfølgelig til tider også anvendt vidensdeling, men har grundet det nogle gange lidt trykkende sociale miljø, lange diskussioner om uoverensstemmelser og misforståelser, ikke anvendt dette så meget som vi reelt have ønsket.

\subsection{Refleksion}
Læringsprocessen har langt fra været en hovedovervejelse under P2. Gruppen har ikke altid været lige god til at få alle med på den samme side. Dette har resulteret i unødvendige misforståelser og tabt viden. Der har manglet en åben dialog omhandlede hvad det enkelte gruppemedlems styrker og svagheder var, sådan at alle videns huller kunne have været belyst. Derfor skal vi i fremtidige projekter være bedre til dette, og tale mere åbent om hvad vi ikke forstår, samt hvad vi synes der ikke fungerer for os.\\
Med hensyn til kurser, skal vi i fremtiden være meget bedre til at spørge underviserne om hjælp, når vi er i tvivl om opgaverne, også selvom vi skal spørge flere gange. Der skal også gøres mere brug af hinandens kompetencer i fremtiden, både i form af kammeratskabslæring, men også blot i form af at spørge om hjælp frit på tværs af gruppensmedlemmer.

\section{Konklusion}
Det kan konkluderes at gruppen ikke altid har være lige velfungerede. Gruppen har været igennem to frafaldende medlemmer, en tilbagevende konflikt i ideologier samt til tider manglende arbejdsmoral. Gruppen oplevede sin største succes da alle forstod hinanden, og da alle have en konkret arbejdsopgave. Trelloen muliggjorde dette i stor stil, og det var tydeligt at i de perioder Trelloen var opdateret, at effektiviteten var på sit højeste. De manglende møder og diskussioner i gruppen, betød dog at Trello kun blev opdateret med jævne mellemrum, det var dog ikke kun alt det betød. Læringsprocessen blev også markant forværet ved denne mangel, hvilket også var med til at skade arbejdsmoralen. Kontakten til vejlederne var mangelfuld, og var præget af at der ikke blev fuldt op på kontrakten.

\newpage
\section{Bilag}
\subsection{Gruppekontrakt}
\label{gruppekontrakt}
\begin{mdframed}[linewidth=0pt,backgroundcolor=lightgray!20,innertopmargin = 0.4cm,innerbottommargin = 0.4cm]
    \textbf{Gruppekontrakt}
\begin{itemize}
    \item 8:15 til 16:15 normal mødetider, medmindre andet er aftalt. 
    \item Kursusgangene er vigtige, vi bruger alt tiden der er sat af til kursusgangene.
    \item Vi er fælles om arbejdsopgaverne i kursusgangene. 
    \item Det er selvfølgelig tilladt at arbejde selv med opgaverne, men gruppen bestræber efter at hjælp hinanden med opgaverne.
    \item Morgenmøde 8:30, der er alle klar til at gå i gang med at arbejde. 
    \item Frokostpause 11.30-12.15, mulighed for alle at få noget at spise og slappe af.
    \item Frokostpause er fri for arbejde, medmindre nogle vil arbejde er dette selvfølgelig tilladt.
    \item Frokostmøde omkring frokostpause, Arbejdsfordeling og vidensdeling.
	Ugemøde om fredagen sidst på dagen. 
    \item Kontakt mellem gruppen foregår på discord. Kontakt efter kl. 20.00 kan ikke forvente svar. Alle tjekker discord hver dag.
    \item Gruppen vil kigge på Latex som skrivemetode til projektet, men ellers vil gruppen benytte Microsoft Teams til at file-sharing og skrivning. 
    \item Skrive til hinanden hvis sygdom eller uheld senest om morgenen inden man møder.
\end{itemize}
\textit{Hvis dette ikke overholdes, vil der blive snakket dette på ugemødet.}
	
\textbf{Vejleder}
\begin{itemize}
    \item Kontakt med vejleder er hele gruppens ansvar og hele gruppen skal være del af beskederne. 
\end{itemize}

\textbf{Forventningsafstemning}
\begin{itemize}
    \item \textbf{Martin}: En fornuftig måde at arbejde sammen med, god tone og godt arbejdsmiljø.
    \item \textbf{Mikkel}: Lære så meget som muligt også i kursusser
    \item \textbf{Daniel}: Vi er her for at lave noget. Motiveret læring, have det sjovt.
    \item \textbf{Benjamin}: Engageret i gruppen, ansvar for gruppen.
    \item \textbf{Christopher}: Have det sjovt i gruppen, fokus på at lære noget. 
	Alle møder op til kursusgangene, vi hjælper hinanden med opgaverne (Vi vil gerne alle bestå)
\end{itemize}
\textit{Skriv hvis man ikke kan nå deadline, vi laver ting til tiden.}

\begin{itemize}
    \item Kontaktoplysninger: Discord, meld sygdom og grunde til for lidt tid til deadline. 
\end{itemize}


\end{mdframed}

\subsection{Vejlederkontrakt}
\label{vejlederkontrakt}
\begin{mdframed}[linewidth=0pt,backgroundcolor=lightgray!20,innertopmargin = 0.4cm,innerbottommargin = 0.4cm]
    \documentclass[11pt]{article}
\renewcommand{\baselinestretch}{1.20} 
\usepackage[utf8]{inputenc}
\usepackage[danish]{babel}\addto\captionsenglish  
{\renewcommand{\bibname}{References}}  
\usepackage[backend=biber,style=numeric,sorting=ynt,block=par]{biblatex}
\usepackage{url}
\usepackage{csquotes}
\usepackage{graphicx}
\usepackage{subfigure}
\usepackage{pdfpages}
\usepackage{caption}
\usepackage{geometry}
 \geometry{
 a4paper,
 total={170mm,237mm},
 left=20mm,
 top=30mm,
 }
\usepackage{fancyhdr}

\pagestyle{fancy}
\fancyhf{}
\rhead{Social hidden Messages 2018}
\lhead{Gruppe: B-125}
\chead{P2 - Vejledningskontrakt}
\rfoot{Page \thepage}

\begin{document}

\section{Vejledermøder}
\begin{itemize}
    \item Alle gruppemedlemmer og vejleder deltager aktivt i vejledermøderne, samt fremmøder forberedt.
    \item Gruppen fremsender en dagsorden for vejledermødet, til vejleder, senest dagen før mødet afholdes.
    \item Gruppens rollefordeling:
    \begin{itemize}
        \item Ordstyrer: Styrer slagets gang og sørger for at dagsorden bliver fulgt (Mikkel).
        \item Referent: Dokumentér mødet (Benjamin).
        \item Gantt Master: Præsentér tidsplanen (Daniel).
    \end{itemize}
    \item Vejledermøder afholdes cirka en gang om ugen og efter behov.
    \item Vejledermøder afholdes, som udgangspunkt, med følgende dagsorden:
    \begin{enumerate}
        \item Hvad er der sket siden sidst. (Evt. Gennemgang af referat).
        \item Gennemgang af fremsendte arbejdsblade (Hvis fremsendt).
        \item Specifikke spørgsmål.
        \item Gennemgå og opdater tidsplan. Gantt og Trello.
        \item Eventuelt.
        \item Dato og tidspunkt for næste vejledermøde.
    \end{enumerate}
    \item Aflysning eller flytning af vejledermøde skal ske senest én time inden planlagt mødestart og så snart som muligt.
    \item Hvis 50\%, eller flere, af gruppens medlemmer er forhindret i at komme til et vejledermøde, skal dette meddeles til vejleder senest én time inden planlagt mødestart.
    \item Vejleder kan, i samarbejde med gruppens medlemmer, aftale om mødet alligevel skal gennemføres.
\end{itemize}

\section{Projektarbejde}
\begin{itemize}
    \item Arbejdsark sendes til vejleder senest en arbejdsdag før afholdelse af vejledermøde. To arbejdsdage giver lidt dybere feedback
    \item Gruppen knytter en kommentar/læsevejledning til hvilken type feedback der ønskes til de fremsendte arbejdsark.
    \item Al mailkorrespondance med vejleder skal sendes "CC." til gruppens medlemmer (itc2b125@student.aau.dk).
    \item Der sendes kun én samlet mail fra gruppen til vejleder indeholdende de spørgsmål gruppen måtte have.
    \item Gruppen skal være enige om mailens indhold inden den afsendes (Review af mindst 1 anden).
\end{itemize}

\end{document}
\end{mdframed}

\end{document}

% Hvordan lærer du bedst, via individuelt arbejde – gruppediskussion – forelæsning etc.?
% Hvordan har I brugt resultaterne af jeres individuelle læringsstilstest?
% Hvilke(n) læringsstrategi-(er) er efter jeres mening bedst i forbindelse med kurser? Hvorfor? Helt klart: Stik ind, stik ud, stik af
% Hvordan hjælper I hinanden med at løse opgaver i kurserne?
% Hvad gør I hvis I ikke forstår en forelæser eller det der står i bogen?
% Hvilke(n) læringsstrategi(-er) er efter jeres mening bedst i forbindelse med projektarbejdet? Hvorfor?
% I hvilket omfang og hvordan stimulerer og fremmer jeres vejledere jeres læreprocesser?