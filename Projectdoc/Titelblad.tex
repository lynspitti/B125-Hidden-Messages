{\selectlanguage{danish}
\aautitlepage{
  \danishprojectinfo{% Rapporten titel:
    Skjult kommunikation i det åbne
  }{% Temaet:
    Netværksbaseret databehandling
  }{% Projektperiode:
    Forårssemestret 2018
  }{% Projektgruppe:
    B125
  }{% Gruppemedlemmer:
    Mikkel Steen Hansen\\
    Daniel Benjamin Vestergaard Jensen\\
    Benjamin Bach Jensen
  }{% Vejleder(e):
    Rasmus Løvenstein Olsen\\
    Julie Rafn Abildgaard (Bi-vejleder)
  }{% Printet kopier / Opslagstal:
    1
  }{% Afleveringsdato:
    \today
  }%
}{% Institut adresse:
  \textbf{Institut for Elektroniske Systemer}\\
  Fredrik Bajers Vej 7\\
  DK-9220 Aalborg Ø\\
  \href{http://es.aau.dk}{http://es.aau.dk}
}{% Abstract:
  This paper explores the possibilities of using public social media platforms to host a secret communication platform, primarily using the method of steganography. Facebook released a report in 2013, stating how they willingly hand out userdata in 80\% of the cases where the american government asked. It is further explained how governments around the world, even the democratic ones, suppress their own people in various different ways, by abusing the power of information. This paper also raises a moral question about the potentially unlawful use cases for a secret communication platform. The technical compromises and difficulties in designing a platform that is built on top of an already existing platform is documented and discussed. The paper concludes that a theoretical solution is possible, but contains a lot of issues, that will affect its effectiveness and appeal for users.
}}
