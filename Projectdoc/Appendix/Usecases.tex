\subsection{Usecases}
\subsubsection{Usecase 1: Login \label{Usecase1_Login}}
\begin{tabular}{@{}p{3.5cm}@{}p{13cm}@{}}
    Usecase funktion & 
    \#1. Login \\
    Beskrivelse & 
    Brugerne logger ind på kommunikationsplatformen på to måder: via platformens offentlige anonyme bots eller via brugeren egen socialemedie konto.\\
    Antagelser & 
    Det antages at:
    \begin{itemize}
        \item Brugeren er i besiddelse af en konto på et af de understøttede socialemedier, såfremt den personlige post funktion skal bruges.
    \end{itemize} \\
    Aktøre & 
    \textbf{Sender} og \textbf{Modtager}, der autentificerer sig via et \textbf{Socialt medie}.\\
    Fremgangsmetode &
    \begin{enumerate}
        \item Brugeren naviger hen til kommunikationsplatformens loginmuligheder.
        \item Brugeren bliver nu stillet to muligheder:
        \begin{enumerate}
            \item Login via egen konto.
            \begin{enumerate}
                \item Brugeren bliver navigeret hen til det sociale medies loginside.
                \item Brugeren udfylder nu sine oplysninger.
                \item Brugeren bliver nu navigeret hen til kommunikationsplatformens velkomstside, med loginoplysningerne fra det sociale medie.
            \end{enumerate}
            \item Login via offentlig konto.
                \begin{enumerate}
                    \item Brugeren bliver navigeret direkte hen til kommunikationsplatformens velkomstside.
                \end{enumerate}
        \end{enumerate}
    \end{enumerate} \\
    Variationer & 
    \begin{itemize}
        \item Loginfunktionen kan foretages ved hjælp af systemets egne bot konti eller ved hjælp af egen konto.
    \end{itemize} \\
    Issues &
    Følgende potentielle problemer med denne usecase:
    \begin{itemize}
        \item Hele loginfunktionen er afhængig af en tredjepart. Hvilket kan have effekt på oppetid og adgang.
    \end{itemize} \\
\end{tabular}
\newpage
\subsubsection{Usecase 2: Send message \label{Usecase2_Send_Message}}
\begin{tabular}{@{}p{3.5cm}@{}p{13cm}@{}}
    Usecase funktion & 
    \#2. Send besked \\
    Beskrivelse & 
    Brugere, der er logget ind i systemet, kan sende skjulte beskeder gennem et socialtmedie. \\
    Antagelser & 
    Det antages at:
    \begin{itemize}
        \item Brugeren har gennemført usecase \#1. "Login".
    \end{itemize}\\
    Aktøre & 
    \textbf{Sender} \\
    Fremgangsmetode &
    \begin{enumerate}
        \item Brugeren naviger hen til kommunikationsplatformens Textformidlings mulighede.
        \item Brugeren vælger et passende forum til sin besked.
        \item Brugeren vælger en tråd, hvor i brugeren vil vedlægge en besked, eller opretter en ny.
        \item Brugeren bliver bedt om at skrive sin besked.
        \item Systemet kryptere brugerens besked, til at ligne et billedes kildekode.
        \item Systemet indopererer beskeden i et billedes kildekode.
        \item Systemet poster beskeden i et offentligt socialt medie.
    \end{enumerate} \\
    Variationer & 
    - \\
    Issues &
    Følgende potentielle problemer med denne usecase:
    \begin{itemize}
        \item Brugeren har mistet sin forbindelse.
    \end{itemize}\\
    Ikke-funktionelle & 
    -
\end{tabular}
\newpage
\subsubsection{Usecase 3: View message \label{Usecase3_View_Message}}
\begin{tabular}{@{}p{3.5cm}@{}p{13cm}@{}}
    Usecase funktion & 
    \#3. Se besked \\
    Beskrivelse & 
    Brugere, der er logget in i systemet, kan se tidligere sendte beskeder, også fra andre brugere. \\
    Antagelser & 
    Det antages at:
    \begin{itemize}
        \item Brugeren har gennemført usecase \#1. "Login".
        \item usecase \#2. "Send message" tidligere har været udført, inden for den aktuelle brugeres tilgang og rettigheder til display af beskeder.
    \end{itemize}\\
    Aktøre & 
    \textbf{Modtager} \\
    Fremgangsmetode &
    \begin{enumerate}
        \item -
    \end{enumerate} \\
    Variationer & 
    - \\
    Issues &
    Følgende potentielle problemer med denne usecase:
    Brugeren har mistet sin forbindelse. \\
    Ikke-funktionelle & 
    -
\end{tabular}
\newpage
\subsubsection{Usecase 4: Delete message\label{Usecase4_Delete_Message}}
\begin{tabular}{@{}p{3.5cm}@{}p{13cm}@{}}
    Usecase funktion & 
    \#4. Slet besked \\
    Beskrivelse & 
    Brugere, der er logget in i systemet, kan slette tidligere sendte beskeder, inden for den aktuelle brugeres tilgang og rettigheder til enkelte beskeder.\\
    Antagelser & 
    Det antages at:
    \begin{itemize}
        \item Brugeren har gennemført usecase \#1. "Login".
        \item usecase #2. "Send message" tidligere har været udført, inden for den aktuelle brugeres tilgang og rettigheder til enkelte beskeder.
    \end{itemize}\\
    Aktøre & 
    \textbf{Sender} \\
    Fremgangsmetode &
    \begin{enumerate}
        \item -
    \end{enumerate} \\
    Variationer & 
    - \\
    Issues &
    Følgende potentielle problemer med denne usecase:
    \begin{itemize}
        \item Brugeren har mistet sin forbindelse.
    \end{itemize}\\
    Ikke-funktionelle & 
    -
\end{tabular}
