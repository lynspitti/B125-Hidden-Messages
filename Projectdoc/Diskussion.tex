\section{Diskussion}
Givet produktets potentiale for anonym kommunikation, samt potentialet for en negativ anvendelse af produktet, er konstruktionen af produktet så overhovedet etisk forsvarligt? I problemanalysen er det blevet vurderet, at det er mere vigtigt at undgå total overvågning end at undgå kriminalitet. Denne vurdering var primært lavet i konteksten af totalitære regimer rundtom i verden. Problemet er blot, at hvis produktet blev tilgængeligt den dag i dag, ville den demokratiske del af verden primært mærke de negative effekter af produktet.\\
For at mindske mængden af misbrug af platformen, kunne det være en mulighed kun at udbrede produktet til betroede individer. Dette ville naturligvis mindske potentialet for anvendelse, men også mindske potentialet for misbrug. Dette vil dog også flytte det etiske spørgsmål væk fra produktet og over på de individer som man betror produktet med. 
Denne brugerbase kontrollering kunne udføres på et par måder. Først og fremmest kunne man holde klient installationen fra det åbne internet, hvormed så få personer som muligt kommer til at få fat i klienten. En anden ting man kunne gøre, er at tvinge de enkelte installationer af klienten til at skulle aktiveres med en unik nøgle genereret af systemet. På denne måde ville man kunne styre antallet af installationer og, igennem fornuftige kommunikationskanaler, kunne man også styre hvilke brugere man får på platformen. Disse to måder vil kunne kombineres for at forøge kontrollen.\\
Selvom denne kontrol måske virker upraktisk, så er den muligvis nødvendig for levedygtigheden af systemet. Dette skal forstås som, at jo større en brugerbase som produktet ville have, jo mere traffik vil det generere på sociale medier. Derved vil det også blive mere sandsynligt, at produktet vil tiltrække uønsket opmærksomhed. Af samme grund kunne det måske også være nødvendigt, at begrænse de enkelte brugeres tilladte opslag på systemet.

% Kan vi få velmenende mennesker til at bruge anonymiteten til noget godt? Eller vil det være for mærkeligt for almindelige mennesker at bruge et forum? (Set i lyset af, at mange er vant til at bruge sociale medier direkte)
% Kan vi tillade os, at løbe risikoen for at ondsindet mennesker bruger systemet?
% Kan vi mindske antallet af ondsindet mennesker igennem indgangskontrol? (For eksempel med unikke aktiveringsnøgler som kun uddeles til betroede individer)
% Er stor skalering overhovedet mulig? Set i lyset af, at sikkerheden primært ligger i ikke at blive opdaget i kommunikationen, vil stor skalering danne meget traffik som igen vil kunne tiltrække uønsket opmærksomhed.

% 1: Produktet kan gøre mere skade end gavn. 
% 2: Kan vi gøre noget for at ændre det? 
% 3: Lille brugerbase er måske nødvendig, givet systemets opbygning.