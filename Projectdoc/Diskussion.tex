\section{Diskussion}
Givet produktets potentiale for anonym kommunikation, samt potentiale for en negativ anvendelse, er konstruktionen af produktet så overhovedet etisk forsvarligt? I problemanalysen er det blevet vurderet, at det er mere vigtigt at undgå total overvågning, end at undgå kriminalitet. Denne vurdering var primært lavet i konteksten af totalitære regimer rundtom i verden. Problemet er blot, at hvis produktet blev tilgængeligt den dag i dag, ville den demokratiske del af verden primært mærke de negative effekter af produktet.\\
For at mindske mængden af misbrug af platformen, kunne det være en mulighed kun at udbrede produktet til betroede individer. Dette ville naturligvis mindske potentialet for anvendelse, men også mindske potentialet for misbrug, samt kunne f.eks. ses anvendt i forbindelse med interne virksomheds værktøjer. Endvidere vil dette også flytte det etiske spørgsmål væk fra produktet, og over på de individer som man betror produktet med. 
Denne brugerbase kontrollering kunne udføres på flere måder. Først og fremmest kunne man holde klient installationen fra det åbne internet, hvormed så få personer som muligt, kommer til at få fat i klienten. En anden ting man kunne gøre, er at tvinge de enkelte installationer af klienten til at skulle aktiveres med en unik nøgle genereret af systemet. På denne måde ville man kunne styre antallet af installationer, og igennem fornuftige kommunikationskanaler kunne man også styre, hvilke brugere man får på platformen. Disse to måder ville selvfølgelig også kunne kombineres for at forøge kontrollen.\\
Selvom denne kontrol måske virker upraktisk, så er den muligvis nødvendig for levedygtigheden af systemet. Dette skal forstås som, at jo større en brugerbase som produktet ville have, jo mere traffik vil det generere på sociale medier. Derved vil det også blive mere sandsynligt, at produktet vil tiltrække uønsket opmærksomhed. Af samme grund kunne det måske også være nødvendigt, at begrænse de enkelte brugeres tilladte opslag på systemet. En potentiel anvendelse, som er etisk forsvarlig, kunne være at en kort liste af betroet journalister indehavende en adgang.
\\\\
I kravspecifikationen blev der opstillet fire krav direkte til sikkerheden som var: \ref{k:stegano}, \ref{k:anonymt}, \ref{k:indstil} og \ref{k:krypto}. \ref{k:stegano} der siger at beskederne skal skjules med steganografi. Dette kunne opfattes som en simpel påhæftning af data i enden af billedets data. Det kunne dog også opfattes som en mere avanceret algoritme, til at placere beskeden "inde mellem" den almene data. Den anden af disse to metoder er langt mere sikker, men også mere kompliceret at udvikle. Hermed påtvinger kravet ikke systemet et bestemt niveau af sikkerhed. Til "proof of concept" vil det derfor også være af den simple type. \\
\ref{k:anonymt} kræver at kommunikationen skal foretages anonym. Dette vil også være tilfældet for kommunikation brugerne imellem. Det skal dog nævnes, at hvis private beskeder i systemet skal være mulige, vil det være nødvendigt at bruge sit private login til det anvendte sociale medie [\ref{privatbesked}]. Dette vil dog påvirke den ellers høje grad af anonymitet, i denne anvendelse, da enkelte konti nu bliver associeret med platformen. \\
Man kunne godt argumentere for, at muligheden for private beskeder, under anvendelse af ens private sociale medie profil, går imod \ref{k:indstil}. Dette krav siger nemlig, at der ikke må forekomme indstillinger, som går på kompromis med brugerens sikkerhed. Dette skal ses i lyset af, at hvis man var under mistanke, så ville det være mere mistænkeligt, hvis ens sociale medie profil havde flere billeder indeholdende krypteret data. \\
Dette leder os til \ref{k:krypto}, som netop siger, at systemet ikke kun må sikres med steganografi. Dette vil være den eneste sikkerhedsfunktion, som stadigvæk ville hjælpe personen i det tidligere tilfælde med et privat login. Hvis beskeden er krypteret vil man i værste fald blot være under mistanke, uden at nogen ville kunne læse beskeden. På den anden side, så vil en krypteret besked, der er gemt i et billede, virke mere mistænkelig.

% Server sikkerhed. 

% Sikkerheds diskussion. Hvad sagde kravspeken? Hvad har vi argumenteret for? Kunne det forstås anerledes? Ville det give mening at bruge en anden fortolkning af sikkerhed.

% Kan vi få velmenende mennesker til at bruge anonymiteten til noget godt? Eller vil det være for mærkeligt for almindelige mennesker at bruge et forum? (Set i lyset af, at mange er vant til at bruge sociale medier direkte)
% Kan vi tillade os, at løbe risikoen for at ondsindet mennesker bruger systemet?
% Kan vi mindske antallet af ondsindet mennesker igennem indgangskontrol? (For eksempel med unikke aktiveringsnøgler som kun uddeles til betroede individer)
% Er stor skalering overhovedet mulig? Set i lyset af, at sikkerheden primært ligger i ikke at blive opdaget i kommunikationen, vil stor skalering danne meget traffik som igen vil kunne tiltrække uønsket opmærksomhed.

% 1: Produktet kan gøre mere skade end gavn. 
% 2: Kan vi gøre noget for at ændre det? 
% 3: Lille brugerbase er måske nødvendig, givet systemets opbygning.