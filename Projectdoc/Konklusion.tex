\section{Konklusion}
Som angivet i problemformuleringen [Se afsnit \ref{problem_formulering}], ville projektet opnå et antal mål. Først og fremmest havde projektet til intention at skabe en kommunikationsplatform, som kunne sikre anonymitet og sikkerhed. Dette ville gøres under anvendelse af steganografi. Nogle af mulighederne for dette er blevet belyst og diskuteret, med det resultat, at der nu forelægger klare idéer for, hvordan det kan gøres. Steganografien i projektet er implementeret under idéen om, at platformens kommunikation skulle skjules blandt alt andet indhold på et givet socialt medie. Dette gøres ved at indlejre den hemmelige besked i et almindeligt udseende billede, og uploade det til en anonym profil på det sociale medie. Andre med applikationen kan så efterfølgende finde beskeden i opslaget, ved at applikationen først finder opslaget og her efter uddrager selve beskeden.\\\\
Platformen forsøger at holde så mange sikkerhed kritiske beslutninger ude af brugernes hænder, både så brugerne ikke skader sig selv, men også så brugeren ikke skader andre. Brugerens anonymitet er blandt andet sikret igennem uploaden på anonyme profiler, samt manglen på login i projektets applikation. Ydermere er det besluttet, at beskederne i systemet skal krypteres, uanset hvordan de indlejres i billederne. Dette er ment som en sikring imod eventuelle systematiske analyser. Dette reducerer angrebsfladen til at være dem, der har adgang til applikationen.\\\\
Et vigtigt aspekt af projektet er, at gøre applikationen anvendelig for en almindelig bruger, der ikke kender til de underliggende metoder så som steganografi og anonyme profiler. Idéen er at skabe en platform, der ikke går på for mange kompromiser, når det kommer til den fundamentale kommunikation, forstået sådan at brugeren ikke skal udføre unormale handlinger, for at fuldføre en kommunikation.\\
Løsningen har dog nogle begrænsninger, i det at løsningen ikke er et selvstændigt system. Det vil for eksempel aldrig kunne køre lige så hurtigt som et selvstændigt system. Der vil også kun kunne anvendes en begrænset mængden af trafik på platformen, før nogen vil opdage denne uregelmæssige traffik, hvilket kan lede til funktionsstop. Med dette menes lukning af f.eks. anonyme brugerkonti, eller hele system servere.\\
Disse negativer gør, at applikationen ikke er perfekt under større udbredelse. På den anden side vil dette sige, at en god anvendelse af systemet kunne være til intern kommunikation journalister imellem, hvor både troværdigheden og antallet af brugere er vurderet af organisationen, som anvender applikationen.