\section{Indledning}
Siden mennesket har kunne kommunikere med hinanden har det også haft behov for at skjule en sandhed eller budskab. Dog har nogle af disse hemmeligheder også senere haft brug for enten at blive gemt, eller delt mellem enkelte individer. Problemet opstår her i at en nedskreven hemmelighed har lettere ved at blive røbet, og at delingen af denne også kan komme til ikke tiltænkte individer.
Derfor har man altid arbejdet i at kunne kryptere, formatere, eller med andre ord skjule, beskeder til alle tænkelige former for brug, lige fra militante strategier til deling af videns-studier. \\
Her skal dog også nævnes at krypto-logi "Læren om det hemmelige", ikke blot er hash koder og data nøgler, men faktisk handler kryptografi om, at skjule informationen i en besked ved at transformere den ind i en forudbestemt algoritme, mens steganografi, til forskel fra kryptografi, skjuler eksistensen af beskeden i stedet for indholdet af beskeden. På den måde kunne beskeder også have været krypteret med et fremmet sprog, både af tale eller tekst, eller ligefrem dannet i tegn og mønstre.\cite{MeningOfCryptography}\\
\noindent
I dag arbejdes der stadig hårdt på at udvikle nye metoder til at skjule meddelelser, faktisk med nutidens udbredelse har behovet for disse metoder aldrig været større. Næsten alle er i dag online, på det ene eller det andet sociale medie, og selv små lande er igennem internettet eksponeret til resten af verdenen. Derfor er det i dag heller ikke kun de geografiske naboer man skal tænke på at skjule sine information fra, som det var i gamle dage, men nu må man også tænke på at skjule selv samme besked fra personer på den anden siden af kloden.
