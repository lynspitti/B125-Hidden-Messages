\subsection{Hemmelig kommunikation}
I dette afsnit vil der blive kigget på, hvordan man tidligere har anvendt kryptografi, for at skabe en basal forståelse for hvordan kryptering, eller kryptografi, er blevet til den teknologi der kendes i dag, samt hvilke fordele eller ulemper der kan tages til videre overvejelser.
\subsubsection{Kryptograferings værktøjer}
Et af de første daterede kryptografi metoder blev opfundet og anvendt af spartanerne ca. 500 år før kristifødsel. \\
\begin{figure}[H]
    \centering
    \includegraphics[scale=1.0]{Projectdoc/Problemanalyse/Illustrationer/scytale.jpg}
    \caption{En spartansk scytale}
    \label{fig:scytale}
\end{figure}
\noindent
Denne metode "Den Spartanske scytale [Se Figur:\ref{fig:scytale}]" åbnede vejen for, hvad vi i dag kender, da denne opfindelse var en af de første, der ikke blot anvendte lokale metoder, så som sprog, men et faktisk værktøj og algoritme til at transponere kendte tekster til kode. Den spartanske scytale var nemlig en cylinder, hvorom man viklede noget at skrive på, herefter skrev man sin besked på de enkelte sider, således når man fjernede cylinderen kunne man ikke forstå sammenhængen, før man havde en cylinder i samme størrelse.\cite{PastCryptography} Dog grundet udvikling vil lige netop dette værktøj i dag ikke vise den bedste effekt, da denne metode danner samme resultat som ved bogstavs flytning, en kode man i dag kan finde i f.eks. et kryds\&tværs hæfte til underholdning, dog findes også andre eksempler på anvendelse af værktøjer, så som kryptograferings maskinen "Enigma" fra Den Anden Verdens krig.

\subsubsection{Hashings algoritmens forgænger - A-K koden}
Ca. 500 år efter den Spartanske scytale opfandt man i Rom, hvad der i dag er den mest kendte og brugte transponerings kryptografi "A-K Koden".\\
A-K Koden går i alt sin simpelhed ud på, at man flytter alfabetet en vis grad, f.eks. i A-K ville "ABCD" skrives "KLMN", eller i A-S ville "ABCD" skrives "STUV".\cite{TheSecretLanguage}\\
Denne krypterings metode har siden sin oprindelse været brugt i et større antal af nyopfundne metoder, ikke kun pga. dens simpelthed og meget store aspekt af kombinationer, men også grundet at den er grundlaget for alt fra simple algoritmer til de sværeste algoritmer. Faktisk er flere af de nu-tids største data Hashings algoritmer også basseret på en længere form af "A-K Koden", og man udregner stadig flere. Dog er A-K Kodens største svaghed også dens udbredelse. Da denne kode i dag er kendt verden over, og efter anvendelse danner et ikonisk rod af usammenhængende bogstaver, der nemt ville genkendes som transformeret data, er denne metode også en af de først anvendte, i forskellige former, hvis en tredjepart skulle de-kryptere.

\subsubsection{Den nyere tids kryptografi}
Faktisk er A-K Koden så alment anvendt, at der kun findes et mindretal af nye og anderledes kryptografi former. Et eksempel på et sådant kunne dog være "Morse Koden" opfundet i 1836. Morse koden anvendte nemlig som andre former end kun tekst, men i stedet også lyd. Lyden blev sendt gennem den datid revolutionerende telegraf og dens netledninger, også i vores tid kendt som telefonnettet.\cite{Telegraphing} Denne nye idé at skjule ikke blot en handling, men også selve dens eksistens som støj, kan siges at have været en større mulighed, hvis ikke dens udførelse havde været så udbredt at den i dag danner kendte ikoniske træk. F.eks. Vil de fleste i dag kunne genkende et "S.O.S".

\subsubsection{De legendariske steganografier}
Foruden alle disse førnævnte daterede og kendte former for kryptografi, kan man selvfølgelig også finde en lang række af ikke daterede, eller ligefrem legendariske kryptografi metoder. Disse kan blandt andet have været anvendt både til handel, men også af f.eks. hjemløse.
\begin{figure}[H]
    \begin{subfigure}{0.5\textwidth}
    \includegraphics[width=0.9\linewidth, height=5cm]{Projectdoc/Problemanalyse/Illustrationer/hobo-glyphs-code.jpg} 
    \caption{The Hobo Code}
    \label{fig:hobocode}
    \end{subfigure}
    \begin{subfigure}{0.5\textwidth}
    \includegraphics[width=0.9\linewidth, height=5cm]{Projectdoc/Problemanalyse/Illustrationer/BurglarsCode.jpg}
    \caption{Da Pinchi Code / The Burglars Code}
    \label{fig:burglarscode}
    \end{subfigure}
    \caption{To af de legendariske kryptografi metoder}
    \label{fig:legendscode}
\end{figure}
\noindent
Et kendt eksempel på en sådan steganografi metode kunne være "The Hobo Code [Se Figur: \ref{fig:hobocode}]", en bestemt del af kryptologien kendt, og anvendt, af hjemløse til at hjælpe hinanden med deres overlevelse\cite{TheHoboCode}. Disse beskeder er aldrig rigtigt blevet dateret, og ingen i dag kender derfor deres præcise oprindelse, men det vides dog stadig at denne kryptologiske metode har været anvendt i flere århundrede.\\ 
Foruden videnen om metodens eksistens gennem årene, vides også at der findes flere andre ligende afarter af denne kendte kode, såsom den nytidiske "Da Pinchi Code [Se Figur: \ref{fig:burglarscode}]". Denne kode bliver alment anvendt af konstruktions arbejdere, men siges også, efter historier, at have været anvendt af indbrudstyve.\cite{DaPinchiCode}

\subsubsection{Anvendelsen}
Dette faktum at der findes flere steganografiske metoder, der dog ikke er særligt kendte, men at man stadig kan anvende dem uden at f.eks. politiet aner uråd, giver et indtryk af at man ville kunne anvende ligende metoder i en nyere teknologi og derved muligvis opnå samme effekt. \\
Man prøver i dag ved alle tænkelige metoder at kryptere forbindelser på nettet, hashe data og oplysninger, eller lige frem at danne sikre VPN tunler mellem sender og modtager. Men alle disse har det tilfældes med A-K koden og Morsing, at de er kendte og synlige, og derfor på et eller andet tidspunkt vil deres sikkerhed blive brudt. Men de legendariske steganografiske metoder har alle den vigtige egenskab at de færreste ligger mærke til dem, selvom de befinder sig lige foran dem, midt på den offentlige gade. Disse egenskaber kunne, muligvis ved nærmere studie, blive anvendt til videregivelse af beskeder på F.eks. De åbne sociale medier, så som Facebook, der allerede kendt for at tilbageholde alle informationer til videregivelse eller studie. Ved andre tilfælde, kunne denne metode måske endda også anvendes til udveksling af information på tværs af modstående lande, så som Rusland og USA.
