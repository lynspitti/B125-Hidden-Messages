\subsection{Den etiske diskussion}
I det foregående afsnit blev nogle typer og grader af statslig censur gennemgået. I dette afsnit vil der fokuseres på nogle etiske overvejelser, i forbindelse med forskellige landes forsøg på at øge lokal sikkerhed. 
\\\\
\noindent
I den vestlige verden er man i højgrad vant til at leve i et samfund, som værdsætter ytringsfrihed og almentgældende menneskerettigheder. På basis af dette har det også været muligt at danne regeringer, som også holder disse værdier højt. På trods af dette har vestlige lande i forskellige grader valgt, at gå på kompromis med basale elementer af privatlivets fred, til fordel for øget sikkerhed mod terrortrusler. Disse er blandt andre videoovervågning i mange vestlige byer, og overvågning af almen internettrafik. Ydermere er efterretningstjenester i stand til at købe information om almindelige personer fra sociale medier såsom Facebook. Den ønskede effekt af disse kompromiser er, at give politi og efterretningstjenester muligheden for, at forhindre terror angreb. Selvom dette lyder ganske fornuftigt, så findes der eksempler på situationer, hvor dette kompromis ikke er helt så simpelt.\\\\
\noindent
Et godt eksempel på dette er de sager hvor FBI har bedt Apple om, at bryde sikkerheden på et af deres egne produkter\cite{FBI/Apple_encryption}. Naivt ville man måske sammenligne dette med at bede en låsesmed om, at åbne en låst dør. Dette er naturligvis inden for hvad politiet forventer at kunne bede om. Problemet er, at analogien i dette tilfælde ikke holder. Det gør den ikke, da man for at bryde cybersikkerhed bliver nød til at skrive noget software, til at kompromittere en bestemt type enhed. Dette stykke software ville så kunne bruges på alle lignende enheder, eller nemt modificeres til dette, og dette er netop problemet. Hvis blot man kunne være sikker på, at denne software aldrig nogen sinde ville blive misbrugt, så ville det principielt være en funktionel løsning. Men det kan man ikke. I kontrast til en fysisk masterkey, hvor fysiske kopier tager tid, penge og ekspertise at producere, så kan en software baseret masterkey kopieres i det uendelige. På den måde går en umiddelbart fornuftig forespørgsel hurtigt over til at have potentielt store og vidt rækkende konsekvenser.\\\\
\noindent
Selv hvis det er muligt at udtænke en fysisk masterkey til brydning af cybersikkerhed, ville man stadigvæk være tvunget til, at stole på den regering som har adgang til den. Dette bringer diskussionen videre til forskelle på demokratiske og diktatoriske lande, såvel som dem i gråzonen. Groft sagt stoler man i den demokratiske verden på, at ens basale menneskerettigheder bliver overholdt, ved hjælp af magtens tredeling. Dette burde sikre, at regeringen ikke agerer uden for lovens rammer. Den almene borger stoler derved på, at denne tredeling holder øje med hinanden, og igennem dette sikrer individets rettigheder. Såfremt dette er tilfældet vil regeringen ikke kunne misbruge de værktøjer, som lovgivningen tillader dem at have, fordi domstolene opererer uafhængigt af regeringen, og sørger for, at de overholder lovgivningen. Samtidigt vil domstolene også sørge for, at der ikke bliver vedtaget nye love, som bryder med samfundets mest fundamentale retningslinjer, altså det enkelte lands grundlov. Men når der arbejdes med cybersikkerhed, er de enkelte lande bare ikke isoleret på samme måde, som de har været tidligere. Med dette menes, at selvom ét lands lovgivning tillader deres regering at være i besiddelse af en masterkey, og måske endda at lovligt tvinge firmaer til at lave masterkeys, så betyder dette ene lands beslutning dermed, at sådan en masterkey vil findes, og dermed også at andre regeringer, eller kriminelle, også kunne få anskaffet samme type udstyr, værende fysisk eller digitalt. Dog skal det siges, at så længe regeringer prøver at tvinge firmaer til at lave masterkeys, så har firmaerne mulighed for at gå til en domstol, hvis de mener det ikke er berettiget. Derfor er det vigtigt at overveje de mulige konsekvenser, ikke blot af kriminelle, men også af hele regeringer, som ikke overholder almene menneskerettigheder.\\
\noindent
I værste fald kunne sådanne forsøg på at øge individets sikkerhed, med hensyn til almen kriminalitet og terror, ende med at yderligere begrænse friheden for store dele af verden, og endda øge cyberkriminalitet markant.