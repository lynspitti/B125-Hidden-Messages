\subsection{Sikring af kommunikationsplatforme}
For at undersøge hvordan moderne kommunikationsplatforme sikrer deres brugere vil to af de største, Facebooks Messenger og WhatsApp, blive diskuteret. I Messenger appen findes en indstilling til at lave en End to End krypteret chat med én anden person. Hvis man ikke kender til funktionens eksistens, så opdager man den måske ikke. WhatsApp har derimod End to End kryptering aktiveret som standard, også for gruppe chats. Forskellen ligger her i, at den informerede bruger kan kommunikere sikkert over begge platforme, men dette kræver dog, at den enkelte bruger selv aktiverer sikkerheden. Dette medfører i praksis, at en del af bruger basen er mindre sikret kun grundet manglen på kendskab til denne sikkerhedsfunktion. Hvorfor er denne kryptering ikke standard på Messenger, når Facebook også ejer WhatsApp\cite{Facebook_WhatsApp_Merger}, hvor det er standard? Dette kunne være for at undgå potentielle problemer med blokering i visse lande\cite{Facebook_security_features}, så som den førnævnte sag med blokeringen af WhatsApp i Brasilien. Samtidigt kunne de ukrypterede samtaler over Messenger være materiale for deres individuelt tilpassede reklamer.
\\
Herefter vil det være naturligt, at betragte WhatsApp som det rigtige valg til privat kommunikation. Problemet med den konklusion er dog tofoldig. Først kan nævnes at man i nogle lande kunne blive mistænkt for at gemme på noget, kun fordi man bruger en platform med bedre sikkerhed end Messenger. Dette er naturligvis aldrig en god ting. For det andet, bør man også overveje hvordan resten af systemet fungerer. Selvom WhatsApp bruger End to End kryptering så har systemet et antal problemer i følge "Electronic Frontier Foundation"\cite{WhatsApp_Security_Concerns}. Ét af disse problemer er et sikkerhedshul som ligger i muligheden for, at lave backups i Google Drev. Denne feature gør det nemt at skifte til et nyt device. Problemet ligger i det faktum at disse backups ikke er krypteret. En konsekvens af dette er, at alle mange brugers chats kan kompromitteres af én brugers ukrypterede cloud backup. Hermed kommer den enkelte brugers valg af sikkerhedsindstillinger til at påvirke andre brugere, som ikke kan gøre noget for at beskytte sig imod dette problem.
\\
Det kan derfor konkluderes at med to af de mest brugte kommunikationsplatforme, så kan awareness på den enkelte brugers part gøre en forskel, men også at andre brugers mangel på samme også kan have en negativ indflydelse på det velvidende indvid. Individets sikkerhed på en kommunikationsplatform burde ikke være afhængigt af andre brugeres valg af indstillinger.