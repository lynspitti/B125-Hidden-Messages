\newpage
\subsection{Internettets oprindelse}
Internettet, som er det fysiske netværk vi bruger til dagligt, havde sin spæde start i 1962 da J.C.R. Licklider beskrev hans "Galactic Network" koncept. Hvad han beskrev var essentielt det som vi i dag kender som internettet. Han forestillede sig nemlig et netværk hvorpå enhver forbunden computer kunne tilgå data og programmer på ethvert givet site. Allerede i 1967 blev planen for ARPANET publiceret og kun to år senere blev de første to nodes koblet på ARPANET. I den efterfølgende måned blev den første host to host besked sendt. Den første offentlige demonstration af ARPANET foregik i 1972, hvilket også var samme år hvor den første elektroniske mail applikation blev lavet. 
Disse to ting viste tilsammen at grundstenene for en helt ny anvendelse af computere, var lige blevet lagt.\cite{brief_history_of_the_internet}

Udviklingen stoppede ikke her, da der blot to år efter blev forslået at koble flere ARPANET lignende netværk sammen i hvad man kunne kalde et internetværk. Dette forslag endte ud i skabelsen af TCP/IP standarden som muligjorde den videre udvidelse af hvad der blev til internettet.\cite{The_development_of_the_internet}

\subsubsection{Hvorfor blev internettet lavet?}
Behovet for et computer baseret kommunikationsnetværk opstod omkring højden af den kolde krig. På denne tid blev der overvejet mange mulige implikationer af, hvad der syntes som uundgåelig, atomkrig. Én af disse overvejelser gik på hvordan man kunne kommunikere over lange afstande, i tilfælde af at telefonnettet blev ødelagt. Hvad der var brug for, var et system som kunne fungere selvom enkeltstående komponenter blev beskadiget. Det var i denne forbindelse at "Galactic Network" konceptet opstod.\cite{The_Invention_of_the_Internet}

Projektet startede derved under det amerikanske forsvars ministerium, og var derfor af militær karakter i starten. Dette begyndte at ændres da ARPANET blev forbundet til stadig flere steder, mange af disse værende universiteter. Denne løbende udvidelse af ARPANET gav anledning til, at projektet nu blev videre udviklet af såvel staten, som private. Selvom det startede med et behov for pålidelig kommunikation i en krise, endte ARPANET med at være værdifuldt blot i effektiviteten af kommunikationen som det tilbød.