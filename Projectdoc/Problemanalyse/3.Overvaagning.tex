\subsection{Overvågning af kommunikation på nettet}
Der vil i dette afsnit blive kigget på hvordan og hvorfor der foregår overvågning og censur på internettet.

\subsubsection{Politisk censur og manipulation}
\label{politisk_censur}
I 2015 fandt Freedom House, som er en uafhængig frihedskæmpende organisation, frem til, at 15 regeringer ud af de 65 undersøgte, havde begrænset befolkningens adgang til diverse sociale medier.\cite{FreedomHouseRapport2016} Tendensen havde i 2016 vokset sig til 24 regeringer.\cite{FreedomHouseRapport2016} Brasilien og Tyrkiet endte 2016's undersøgelse med, at blive to af de mest bemærkelsesværdige lande, da de begge gik et betydeligt skidt tilbage på deres respektive frihedsskalaer, netop på grund af deres magtanvendelse overfor diverse sociale medier. Brasilien gik fra kategorien "Free" til "Partly Free", da brasilianske domstole indførte en midlertidig blokering af opkald- og tekstkommunikations tjenesten WhatsApp. WhatsApp nægtede nemlig at udlevere brugerdata til bevismateriale. Tyrkiet gik fra kategorien "Partly Free" til "Not Free" efter masseblokeringer af diverse sociale medier, og efterfølgende forfølgelse af borgere, der kritiserede Tyrkiets regering.\cite{FreedomHouseRapport2016}

Det kan virke som om at de sociale medier kun er til befolkningens fordel, med eksempler på organisering af store demonstrationer og muligheden for at dele beretninger om uretfærdighed fra hele kloden. Virkeligheden er dog at regeringerne lytter med. Regeringer er blevet bedre til at manipulere og sprede deres politiske agenda på de sociale medier end borgerne og aktivisterne.\cite{SocialHelpDictators} Det ses blandt andet i Freedom Houses rapport fra 2017, som fandt mindst 18 regeringer der, med hjælp af sociale medier, aktivt har spredt misinformation der manipulerede med valgenes valg, heriblandt det demokratiske land USA.\cite{FreedomHouseRapport2017} Et af de eksempler på hvorfor USA er at finde på listen over disse land, er blandt andet Trumps kampagne og deres brug af Cambridge Analyticas tvivlsomme brug af data [Ses nærmere i afsnit \ref{national_privatliv}].\cite{Cambridge_Analytica_Zuckerberg}
Dertil kommer forgangsprogrammer som den amerikanske efterretningtjeneste NSAs PRISM, som tillader dommerkendelsesløs aflytning af blandt andet email, netopkald og kommunikation over sociale medier,\cite{PRISM} der er med til at skubbe grænserne for regeringers kontrollering af deres befolkning. Politisk censur og manipulation af internettet hører ikke blot diktatorer til, men er et globalt fænomen der ikke viser de store tegn på at begrænse sig.\cite{FreedomHouseRapport2017}

\subsubsection{National sikkerhed kontra privatliv}
\label{national_privatliv}
Når privatlivets fred diskuteres kommer to grundlæggende holdninger ofte op. De to yderpunkter er, "Overvågning er ikke et problem hvis du ikke har noget at skjule" og "Ingen skal have ret til at vide hvad jeg laver". Situationen er så at begge yderpunkter har problemer. Hvis individet f.eks. var garanteret at ingen agentur kunne overvåge, undersøge, eller blot spørge til, aktiviteter så ville det ikke være muligt at beskytte almene borgere mod kriminalitet og terror. På den anden side vil et samfund i den anden ende af spektrummet også være i stand til at bruge informationen imod borgernes interesse. 
Et eksempel på denne form for misbrug af data er sagen med Cambridge Analytica, som har benyttet data fra 87 millioner Facebook brugere til bl.a. at hjælpe Trumps valgkampagne i 2016.\cite{Cambridge_Analytica_Zuckerberg} Denne type anvendelse af data kunne have omfattende konsekvenser for demokratiets grundlag, hvis enhver siddende regering kunne bruge deres borgeres information i lignende Big Data baseret valgkampanger. 

Et andet eksempel på konflikten mellem national sikkerhed og privatlivets fred er bagdøre til krypterede systemer. På den ene side ville denne bagdør blive brugt af regeringer til at fremme efterforskningen af kriminelle. På den anden side kan sådanne bagdøre også anvendes af kriminelle for finansielle gevinster eller af regeringer som vil undertrykke deres befolkninger. FBI og Apple har været i netop sådanne situationer hvor Apple ikke har ville lave en generel bagdør til FBI. Dette valg traf de for at beskytte deres brugere.\cite{FBI/Apple_encryption}

Essensen er, at de fleste mennesker er enige om, at samfundet burde have sikkerhedsforanstaltninger imod kriminalitet og terrorisme. Samtidigt vil borgere gerne have en del privatliv. Både fordi det er ubehageligt at blive overvåget i alle aspekter af ens liv, men også fordi der er potentielt store konsekvenser, hvis ikke dette er tilfældet.

% Det egentlige spørgsmål kommer tilbage til: Hvor meget overvågning er for meget? Dette projekt vil tage udgangspunkt i den overbevisning at der ikke ønskes et samfund hvor man ikke har ret til et privatliv. 

% Udviklingen har igennem længere tid været sådan, at regeringer har fundet nye måder at overvåge befolkningen på, og efter noget tid, er det blevet fastslået ved lov, i hvilken grad, og i hvilke tilfælde, at denne type overvågning skal være lovlig. I de seneste år har udviklingen af teknologi dog været meget hurtigere end retssystemet. Det har derfor været svært at regulere disse nye teknologier på passende vis. Problemet er dog mere omfattende når stater uden veletableret retssystemer betragtes. I disse stater kan regeringer bruge disse nye teknologier til at fremme deres egne formål. Borgerne i disse stater har så adopteret metoder til at undgå overvågning, så som VPN tjenester og TOR netværket. Regeringer kan så som modsvar blokere VPN tjenester i en stor udstrækning. At blokere borgere fra at bruge TOR netværket er dog ikke lige til, da al trafikken er krypteret og tager tilfældige veje igennem netværket. Problemet ved dette er, at meget store mængder trafik kan blive betragtet som værende forsøg på hemmelig kommunikation.