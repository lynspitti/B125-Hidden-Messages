\newpage
\subsection{Statslig censur af internettet}
Den global internet frihed er på syvende år i træk stadig dalende \cite{FreedomHouseRapport2017}, lyder det fra den uafhængige frihedskæmpende organisation Freedom House i deres årlige rapport "Freedom on the Net". Freedom House har til formål at fremme frihed og demokrati rundt omkring i verden, blandt andet ved at fortage dybdegående analyser og ved at fortale menneskerettigheder \cite{FreedomHouseAbout}. Den årlige "Freedom on the Net" rapport afdækkede i de to nyeste udgivelser (2016 og 2017) internetfriheden for mindst 87\% af verdens internet benyttede befolkning og giver hvert år en resumering af hvilke trends der havde størst betydning for deres samlede konklusion. 

Hvert af de 65 lande der årligt bliver afdækket, får tildelt sig en numerisk score alt efter hvor god eller ringe deres internet frihed har været. Denne score går fra 0 til 100 og er ydermere opdelt i 3 overordnet kategorier: Free (0 - 30 points), Partly Free (31 - 60) og Not Free (61 - 100). En score er også bestemt ud fra 3 overordnet kategorier: Forhindringer for adgangen til nettet (Giver 0 - 25 points), restriktioner på indholdet (0 - 35) og overtrædelser af brugernes rettigheder (0 - 40) \cite{FreedomHouseRapportMethodology}.

\subsubsection{Censur af kommunikationskanaler}
I 2016 udgivelsen af rapporten blev den stigende magtmisbrug af de sociale medier samt andre kommunikationskanaler belyst. I 2015 fandt Freedom House frem til at 15 regeringer ud af de 65 undersøgte havde begrænset befolkningens adgang til diverse sociale medier, i 2016 havde dette tal stiget til 24. Brasilien og Tyrkiet endte 2016's undersøgelse med at blive to af de mest bemærkelsesværdige lande, der begge gik et betydeligt skidt tilbage på deres respektive frihedsskalaer, netop på grund af deres magtanvendelse overfor diverse kommunikationskanaler. Brasilien gik fra kategorien "Free" til "Partly Free" da brasilianske domstole indførte en midlertidig blokering af opkald- og tekstkommunikations tjenesten WhatsApp, da WhatsApp nægtede at udlevere brugerdata til bevismateriale. Tyrkiet gik fra kategorien "Partly Free" til "Not Free" efter masseblokeringer af diverse kommunikationskanaler og efterfølgende forfølgelse af borgere der kritiserede Tyrkiets regering. 

Den mest hyppige årsag til regeringers, især autoritære regimers, restriktioner af diverse kommunikationskanaler er for at holde deres befolkninger i skak. Disse restriktioner sker blandt andet når en regering kritiseres, når der mistænkes korruption eller under anden aktivisme. Da disse kommunikationskanaler er så gode til at for samlet folk og organisere protester og lignede, vil visse regeringer bruge deres magt på undertrykkelse. Det er dog ikke kun aktivisme nogle af disse autoritære regimer undertrykker, også homoseksuelle fællesskab, satire, religiøse og modstående politiske overbevisninger bliver undertrykt \cite{FreedomHouseRapport2016}.

\subsubsection{Angreb mod kryptering}
Moderne efterforskning inkludere i højere grad de forskellige kommunikationskanaler når kriminalitet og terroristisk aktivitet skal bekæmpes. Smartphone, computere og online services er i vor tid tætpakket med avanceret kryptering, både for at beskytte virksomhedens brugere men også virksomheden selv. Denne kryptering så myndighederne i blandt andet Kina, Ungarn, Rusland, Thailand, Storbritannien og Vietnam gerne gradbøjet, som led i individuelle lov ændringer indefor området. Disse love kan kræve at virksomheder udlevere backdoor krypteringsnøgler til myndighedernes efterretningstjenester. Dette kan udgøre en kolossal risiko for alle brugerene af det pågældende medie, især de brugere der har brug for en sikker kanal til journalistisk eller aktivisme. Back

VPN'er og Backdoors
\cite{FreedomHouseRapport2017}
