\subsection{Moderne kommunikation}
I det foregående afsnit blev udviklingen af hemmelig kommunikation gennemgået og en bestemt metode, steganografi, blev fremhævet som et muligt værktøj. I dette afsnit vil det blive undersøgt hvorfor og i hvilken udstrækning, at en ny type af hemmelig kommunikation skulle være nødvendig.

\subsubsection{Censur og overvågning af kommunikationsplatforme}
Den globale internet frihed er på syvende år i træk stadig dalende\cite{FreedomHouseRapport2017}, lyder det fra den uafhængige frihedskæmpende organisation Freedom House i deres årlige rapport "Freedom on the Net".
I 2016 udgivelsen af rapporten, blev den stigende magtmisbrug af de sociale medier, samt andre kommunikationskanaler, belyst. I 2015 fandt Freedom House frem til, at 15 regeringer ud af de 65 undersøgte, havde begrænset befolkningens adgang til diverse sociale medier, i 2016 havde dette tal forøget sig til 24. Brasilien og Tyrkiet endte 2016's undersøgelse med, at blive to af de mest bemærkelsesværdige lande, da de begge gik et betydeligt skidt tilbage på deres respektive frihedsskalaer, netop på grund af deres magtanvendelse overfor diverse sociale medier. Brasilien gik fra kategorien "Free" til "Partly Free", da brasilianske domstole indførte en midlertidig blokering af opkald- og tekstkommunikations tjenesten WhatsApp. WhatsApp nægtede nemlig at udlevere brugerdata til bevismateriale.\\ 
Tyrkiet gik fra kategorien "Partly Free" til "Not Free" efter masseblokeringer af diverse sociale medier, og efterfølgende forfølgelse af borgere, der kritiserede Tyrkiets regering.\cite{FreedomHouseRapport2016}\\

%Den mest hyppige årsag til regeringers, især autoritære regimers, restriktioner af diverse sociale medier, er for at holde deres befolkninger i skak.
%Disse restriktioner sker blandt andet når en regering kritiseres, når der mistænkes korruption eller lignede.
%Gennem tiden har utilfredse bevægelser kunne samle sig og kæmpe for hvad de troede på, og dette har ført til blandt andet væltede monarker, arbejdsgivere og regimer. 
Det ses stadig i dag hvordan utilfredse befolkninger rundt om i verden kæmper mod uretfærdighed, f.eks under Det Arabiske Forår hvor at en række arabiske lande gjorde op med deres diktatoriske ledelse.\cite{ArabiskeForaar} Under disse oprør, især i nyere tid, har de sociale medier været uvurderlige i at få befolkningsgrupper engageret og oplyst til at ville kæmpe for deres respektive sag. Da disse sociale medier er gode til at få samlet folk, og organisere protester og lignede, vil visse regeringer bruge deres magt på undertrykkelse. 
%Det er dog ikke kun aktivisme, nogle af disse autoritære regimer undertrykker også homoseksuelle fællesskab, satire, religiøse- og modstående politiske overbevisninger.
Disse udsatte grupper har i høj grad brug for at de kommunikationskanaler de benytter beskytter deres kommunikation og oplysninger, sådan de ikke falder i de forkerte hænder.

\subsubsection{Sikkerhed og privatlivet i fare}
En forøgelse i individers private sikkerhed vil også komme kriminelle og terrorister til gavn, da det vil gøre det nemmere for dem at skjule deres intentioner og handlinger.
% Moderne efterforskning inkluderer i højere grad de forskellige kommunikationskanaler, når kriminalitet og terror aktivitet skal bekæmpes. Smartphone, computere og online services er i vor tid tætpakket med avanceret kryptering, både for at beskytte virksomhedens brugere, men også virksomheden selv. Denne kryptering så myndighederne, i blandt andet Kina, Ungarn, Rusland, Thailand, Storbritannien og Vietnam, gerne gradbøjet, som led i individuelle lov ændringer indefor området. Disse love kan blandt andet kræve at virksomheder udleverer backdoor eller bagdørs krypteringsnøgler til myndighedernes efterretningstjenester. Dette kan udgøre en kolossal risiko for alle brugerene af det pågældende medie, især for de brugere der har brug for en sikker kanal til journalistik, aktivisme osv.\cite{FreedomHouseRapport2017} Det åbner op for at visse regeringer har muligheden for at misbruge brugerne på de sociale mediers data udenfor et efterforsknings miljø, hvilket også er virksomhedernes største frygt i forbindelse med disse bagdøre.\\
Nogle stater har endda diskuteret delvis blokering af VPN tjenester for at gøre det sværere for kriminelle at gemme sig. Dette ville dog også påvirke individer der blot vil diskutere landets politiske klima, eller andre emner som de enkelte stater har sat sig imod.

Et eksempel på konflikten mellem national sikkerhed og privatlivetsfred er bagdøre til krypterede systemer. På den ene side ville denne bagdør blive brugt af regeringer til at fremme efterforskningen af kriminelle. På den anden side kan sådanne bagdøre også anvendes af kriminelle og regeringer som vil undertrykke deres befolkninger. FBI og Apple har været i netop sådanne situationer hvor Apple ikke har ville lave en generel bagdør for, at beskytte deres brugere.\cite{FBI/Apple_encryption}

% Et godt eksempel på dette er de sager hvor FBI har bedt Apple om, at bryde sikkerheden på et af deres egne produkter\cite{FBI/Apple_encryption}. Naivt ville man måske sammenligne dette med at bede en låsesmed om, at åbne en låst dør. Dette er naturligvis inden for hvad politiet forventer at kunne bede om. Problemet er, at analogien i dette tilfælde ikke holder. Det gør den ikke, da man for at bryde cybersikkerhed bliver nød til at skrive noget software, til at kompromittere en bestemt type enhed. Dette stykke software ville så kunne bruges på alle lignende enheder, eller nemt modificeres til dette, og dette er netop problemet. Hvis blot man kunne være sikker på, at denne software aldrig nogen sinde ville blive misbrugt, så ville det principielt være en funktionel løsning. Men det kan man ikke. I kontrast til en fysisk masterkey, hvor fysiske kopier tager tid, penge og ekspertise at producere, så kan en software baseret masterkey kopieres i det uendelige. På den måde går en umiddelbart fornuftig forespørgsel hurtigt over til at have potentielt store og vidt rækkende konsekvenser.\\\\

% En anden sikkerhedsforanstaltning, som diverse regeringer, herunder Kina, Rusland og Egypten\cite{FreedomHouseRapport2017}, ønsker at regulere, er VPN'er eller Virtual Private Networks. En VPN beskytter brugerens privatliv og sikkerhed på nettet, ved at fungere som et ekstra led mellem brugerens enhed og internettet [Se Figur: \ref{fig:vpn}]. Denne forbindelse er krypteret, som betyder at brugerens data og handlinger er sikret.\cite{VPNInfo} Derudover kan en VPN virtuelt skifte den geografiske position af en given enhed, og dermed undgå eventuelle regeringers censur, af for eksempelvis internationale nyhedsbureauer, eller sociale medier. Selvom VPN kan blive brugt til kriminel aktivitet, så bruger de fleste den til godsindet formål, såsom: at værne om privatlivet, forblive informeret via censureret medier, eller blot som et led i deres internationale arbejde. De regeringer der ønsker at regulere VPN'er, vil dog ikke blokerer VPN'erne helt, da de netop bruges af regeringens ansatte, som et essentielt redskab. Regeringerne ønsker en proces, hvor at den enkelte VPN udbyder skal autoriseres til at operere i et given land, ved et sæt defineret brugstilfælde, resten af udbyderne vil blive blokeret af regeringen.\cite{FreedomHouseRapport2017}

%Den verdens dækkende udbredelse af internettet, og dets anvendelighed til kommunikation, betød ikke blot en løsning af manges ønsker på forbindelse og deling. Det global samfund havde nu fundet en forholdsvis åben dør, kun få klik værk fra alle de tænkelige informationer de kunne drømme om. Denne magt internettet pludselig fik over menneskers data, betød jo selvfølgelig også en vækst af magtmisbrug, over det internet der ellers var tiltænkt at skulle være åbent og frit.
%Freedom House har til formål at fremme frihed og demokrati rundt omkring i verden, blandt andet ved at fortage dybdegående analyser, og ved at være fortaler for menneskerettigheder.\cite{FreedomHouseAbout} 
%De nyeste årlige "Freedom on the Net" rapporter fra 2016 og 2017, der dækker mindst 87\% af verdens internet benyttende befolkning, giver hvert år et resumé af hvilke trends der havde størst betydning for deres samlede konklusion. \\
%Hvert af de 65 lande der årligt bliver afdækket, får tildelt sig en numerisk score, alt efter hvor god eller ringe deres internet frihed har været. Denne score går fra 0 til 100, og er opdelt i 3 overordnet kategorier: Free, Partly Free og Not Free [Se Tabel: \ref{fig:freedomscale}].\\ 
%En score er bestemt ud fra 3 overordnet kriterier: Forhindringer for adgangen til nettet, restriktioner på indholdet og overtrædelser af brugernes rettigheder [Se Tabel: \ref{fig:freedomscale2}].\cite{FreedomHouseRapportMethodology}