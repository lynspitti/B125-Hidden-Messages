%Systemet beskrevet ved [Figur \ref{fig:sysdiagram}] er baseret på en brugers aktive handlinger, og %kan derved styres fra et centralt sted. Her tales om enten brugerens egen enhed, eller en central %server, der ikke fortager andre handlinger end at responderer klart på en brugers kommandoer.

%I systemets handlings diagram ovenfor [Figur \ref{fig:sysdiagram}] ses produktets overordnede handlingsflow. Diagrammet er ikke en endelig eller færdig udarbejdelse, men skabt til at give et letforståeligt overblik over systemets indhold, nødvendige resurse behov, samt mulige anvendte teknologier. Ud fra diagrammets flow, med start fra den sorte cirkel i toppen, kan der som et eksempel ses, at systemet anses for at skulle have tilgang til en Database, indeholdende mulige Bot associeringer til en brugeres login, eller en tilgang til det anvendte sociale medies API for samme.\\
%I dette eksempel er f.eks. ikke videre beskrevet hvordan, eller om systemet overhovedet skal, håndtere denne databases placering eller tilgang, efter en eventuel spærring fra tredjeparter, der f.eks. kunne anse systemet for at understøtte maliciøse aktiviteter.\\
%Efter denne login aktivitet ses hovedmenuen, hvorfra brugerne kan vælge at f.eks. tilgå systemets indstillinger, danne en ny besked, eller hente tidligere beskeder og tråde. Denne lægger også op til et andet dilemma omhandlende hvordan systemet, uden at skabe en større række af mistænkelige forespørgsler, skal kunne genkende, hente, og sammekæde flere beskeder på op til flere mulige brugere kontoer, hvilket efter simple normale requests vil kræve op til måske flere hundrede transmissioner per request.\\
%Senere i dette afsnit vil disse føromtalte dilemmaer såvel som, teorier og tanker blandt andre blive videre redegjort for, med henblik på at skabe et endeligt oplæg til en række tekniske læsningsforslag.