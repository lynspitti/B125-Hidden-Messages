\section{Problemafgrænsning}
Gennem problemanalysen er det blevet gennemgået og diskuteret, hvordan behovet for at skjule informationer fra uvedkommende, altid har været efterspurgt, og stadig er nødvendig den dag i dag. Det er et relevant emne i dag at bevare sine personlige informationer, samtidig med at kunne anvende internettet i sikkerhed for overvågning og censur. Dette er blandt andet et problem for brugere af internettet i lande med lavere internetfrihed så som Tyrkiet og Kina.\cite{FreedomHouseRapport2017}\\
\noindent
Data sikkerhed er i dag blevet en industri standard for mange virksomheder, der forstår vigtigheden ved at få sine kunder og brugere til at føle sig sikre. Dette resulterer dog i et øget fokus fra blandt andet kriminelle og myndigheder på disse sikkerheds metoder, og hvordan de kan penetreres.
\\\\
Det er gennem problemanalysen blevet antydet at den tekniske sikring af kommunikation, ikke blot kan opnås ved at kryptere en besked, men også ved at skjule beskedens eksistens. Selvom en krypteret besked er ulæselig for den uvedkommende, kan denne stadig se at der er tale om en form for hemmelig kommunikation, som i sig selv kunne give anledning til blokering fra bl.a. regeringer og lignende. Nogle af de grupper der kan være nødsaget til at kommunikere i al hemmelighed er blandt andet aktivister, journalister og hackere, både de gode og ondesindet. Disse grupper er særligt udsatte alt efter hvad de laver og hvor i verden de befinder sig.\cite{FreedomHouseRapport2017}
\\\\
Disse grupper kunne derfor drage gavn af en internetbaseret kommunikationsplatform, som kan undgå overvågning, som ikke vil blive opfattet som værende et forsøg på hemmelig kommunikation. 
Da overvågningen på internettet kan antages at være altdækkende, bliver en oplagt mulighed for at undgå overvågning, at anvende steganografi.