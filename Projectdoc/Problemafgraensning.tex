\section{Problemafgrænsning}
Gennem problemanalysen er det blevet gennemgået og diskuteret, hvordan behovet for at skjule informationer fra uvedkommende, altid har været efterspurgt, og stadig er nødvendig den dag i dag, også selv på det ellers så udviklede, brugte og kendte internet. Det er et stort og vigtigt emne i dag at bevare personlige informationer, samtidig med at kunne anvende internettet i fred for overvågning og censur. Kryptering er blevet en industri standard for mange virksomheder, der forstår vigtigheden ved at få sine kunder og brugere til at føle sig sikre. Dette resulterer dog i et øget fokus fra blandt andet kriminelle og myndigheder på kryptering, og hvordan denne kan penetreres.
Kryptering er et emne der bliver forsket indenfor i stor stil, for at skabe nye og bedre krypteringsmetoder. Forskningen i den private sektor sker med sådan et tempo at selv de militære sikkerhedsprotokoller anvender privat kendt sikkerhedskryptering.\cite{ForsvaretsSikkerhed}
\\\\
Det er gennem problemanalysen gruppens opfattelse at sikker kommunikation, ikke blot drejer sig om at kryptere en besked, for at hindre en uvedkommende i at modtage den, men også et spørgsmål om synlighed. Selvom en kryptering for den uvedkommende er ulæselig, kan vedkommende stadig se at der er tale om en hemmelig kommunikation, som i sig selv kunne give anledning til blokering fra bl.a. regeringer. I dag findes mange grupper hvor en samtale eller anden korrespondance, for andre kunne virke ganske harmløs og uskyldig, mens blandt de interne faktisk kan have skjult en større hemmelig besked, lige for næsen af alle.
\\\\
Denne ide, at udsende en krypteret hemmelig besked lige for næsen af alle, uden deres opmærksomhed, mener gruppen ville kunne anvendes i en større sammenhæng, lige fra at opretholde friheden på nettet, men også til anvendelse i en mere statslig forstand, som f.eks. militæret.
